\input{header.tex}


\begin{document}

\maketitle

Dieser Text ist unter dieser \href{http://creativecommons.org/licenses/by-nc-sa/4.0/}{Creative Commons} Lizenz veröffentlicht.

\textcolor{red}{Ich erhebe keinen Anspruch auf Vollständigkeit oder Richtigkeit. Falls ihr Fehler findet oder etwas fehlt, dann meldet euch bitte über den Emailkontakt.}

\tableofcontents


\newpage



\section{Aufgabe 1 - Astronomie}

\begin{figure}[h]
	\centering
	\includegraphics[scale=0.6]{A1_1.jpg}
\end{figure}

\section{Aufgabe 2 - Sphärische Abberation}

\subsection*{a)}

Was ist die Ursache der shärischen Abberation?

\subsection*{b)}

?

\subsection*{c)}

Zeichnen sie den Strahlenverlauf durch eine Linse unter Berücksichtigung der sphärischen Abberation.

\subsection*{d)}

Was ist die älteste Methode um die sphärische Abberation zu korrigieren und was ist ihr Nachteil?

\subsection*{e)}

Gibt es heute bessere Methoden?


\section{Aufgabe 3 - Brechung}

Man hat ein \textbf{gleichseitiges} Glasdreieck, das von einem Lichtstrahl symetrisch durchlaufen wird. $n_{Glas} = 1,6$ und $n_{Luft} = 1$. Der Lichtstrahl fällt parallel zur Grundseite ein.


\begin{figure}[h]
	\centering
	\includegraphics[scale=0.5]{A3_1.jpg}
	\caption{Skizze}
\end{figure}

\subsection*{a)}

Kommt es zur Totalreflexion, wenn sich an Punkt A Luft befindet?

\subsection*{b)}

Kommt es zur Totalreflexion, wenn sich an Punkt A Wasser ($n_{Wasser} = 1,334$) befindet?

\newpage

\section{Aufgabe 4 - Sterne und Teleskope}


Ein Stern hat den Durchmesser $\unit[2,38 \cdot 10^6]{km}$ und Abstand $\unit[8,6]{Lj}$.


\subsection*{a)}

Die Sterne nahe der Sonne haben alle die gleiche Größe, aber unterschiedliche Helligkeiten. Woran liegt das?


\subsection*{b)}

Du hast ein Teleskop dessen Linse eine Brennweite $\unit[25]{m}$ beträgt. Berechne die Bildgröße des Sterns.


\subsection*{c)}

\begin{figure}[h]
	\centering
	\includegraphics[scale=0.5]{A4_1.jpg}
\end{figure}

\section{Aufgabe 5 - Besselversuch}

Alles zum Besselversuch. Aufbau, Funktion, Methode, Berechnung der Brennweite der vermessenen Linse.





















\end{document}